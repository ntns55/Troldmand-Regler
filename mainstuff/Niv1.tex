\chapter*{Niveau 1}
\addcontentsline{toc}{chapter}{Niveau 1}
Som troldmand eller kvinde kræves der meget studie. De fleste starter som en bibliotekar eller bruger år på at lære matematik for at kunne fremkalde den mindste gnist. Dette er det sted du befinder dig i din rejse, det første skridt mod utrolig magt. 

\begin{table}[H]
    \centering
    \begin{tabular}{|p{0.50\textwidth}|p{0.25\textwidth}|}
    \rowcolor{cerulean!80}\hline
        Evne navn & Pris i XP \\\hline
        Læse/Skrive Elvisk & 1\\\hline
        Læse/Skrive Magi & 1\\\hline
        Magiske Studier & 1\\\hline
        Troldmandsmagi Niv. 1 & 1\\\hline
    \end{tabular}
\end{table}

\section*{Evne beskrivelse}
\addcontentsline{toc}{section}{Evne beskrivelse}

\input{../Evne-Ordbog/Læse og skrive/Læse og skrive Elvisk}

\input{../Evne-Ordbog/Læse og skrive/Læse og skrive Magisk}

\subsection*{Magiske Studier}\addcontentsline{toc}{subsection}{Magiske Studier}
Troldmanden leder i gamle bøger og magiske teorier for at finde svar på et spørgsmål der relatere sig til magi.\\
Troldmanden kan stille et spørgsmål til arrangørerne angående magiske egenskaber i verden, som han vil få svar på.

\subsection*{Troldmandsmagi Niv. 1}\addcontentsline{toc}{subsection}{Magiske Studier}
Troldmanden kan nu kaste niveau 1 magier fra deres sti. Se mere information under kapitlet 'Magi som Troldmand' under sektionen 'Niveauerne'. \\
Derudover får du en ny titel som afhænger af hvilken sti du har valgt.\\
\begin{itemize}
    \item Dæmonologen får titlen \textbf{Kætter}
    \item Elementalisten får titlen \textbf{Åndsløs}
    \item Mentalisten får titlen \textbf{Lærling}
    \item Nekromantikeren får titlen \textbf{Orm}
\end{itemize}