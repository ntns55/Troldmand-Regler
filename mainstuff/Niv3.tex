\chapter{Niveau 3}
At kaste med magi ligger i din natur. Du lader altid til at have forstand på, hvad der sker omkring dig, og det er ikke unormalt at du bruger mere tid med bøger end med andre levende væsner. Når det så er sagt, så kan du nemt besejre en hvilken som helst kriger ved at pege på dem.

\begin{table}[H]
    \centering
    \begin{tabular}{|p{0.50\textwidth}|p{0.25\textwidth}|}
    \rowcolor{cerulean!80}\hline
        Evne navn & Pris i XP \\\hline
        Ekstra Mana Niv. 2 & 2\\\hline
        Magisk Fælde & 1\\\hline 
        Troldmandsmagi Niv. 3 & 2\\\hline
    \end{tabular}
\end{table}
\section*{Evne beskrivelse}
\addcontentsline{toc}{section}{Evne beskrivelse}

\input{../Evne-Ordbog/Ekstra Mana/Ekstra Mana Niv. 2.tex}


\subsection{Magisk Fælde}
Du kan ligge en magisk effekt du kender, som maks varer 30 sekunder, på en lås. Denne går af når låsen dirkes op. Denne effekt vil kun kunne gå af en gang før denne evne skal bruges igen. Du bliver ikke påvirket af fælden, hvis låsen låses op med en nøgle. For at bruge denne evne skal du bruge 1 mana krystal per niveau af den magi du sætter på låsen. Denne fælde forsvinder når den er brugt 1 gang.


\subsection{Troldmandsmagi Niv. 3}
Troldmanden kan nu kaste niveau 3 magier fra deres sti. Se mere information under kapitlet 'Magi som Troldmand' under sektionen 'Niveauerne'. \\
Derudover får du en ny titel som afhænger af hvilken sti du har valgt.\\
\begin{itemize}
    \item Dæmonologen får titlen \textbf{Helvedes Betvinger}
    \item Elementalisten får titlen \textbf{Betvinger}
    \item Mentalisten får titlen \textbf{Medium}
    \item Nekromantikeren får titlen \textbf{Dødens Tjener}
\end{itemize}