\chapter{Niveau 4}
Magisk energi ligger i dit blod, og en enkelt af dine øjenvipper har mere magt end de fleste hære. Du kan skabe og destruere blot med tankens kraft, og din visdom er noget som er indlært efter mange års studie. Dit største problem er, at almindelige væsner virker så ordinære i sammenligning med dig.\\
\textbf{Ærkemager} er titlen givet til de få personer, som vælger at lære det ypperste indenfor magien. De kan altid finde noget at bedrive tiden med, om det er et studie i, hvor meget en svamp gror på et år eller lave en ny ildkugle, designet efter en drages flammer er ikke til at vide, og det er kun Ærkemageren, som ved hvilke af disse to er mest destruktivt.\\

\begin{table}[H]
    \centering
    \begin{tabular}{|p{0.50\textwidth}|p{0.25\textwidth}|}
    \rowcolor{cerulean!80}\hline
        Evne navn & Pris i XP \\\hline
        Ekstra Mana Niv. 3 & 3\\\hline
        Lav Manakrystal & 2\\\hline 
        Troldmandsmagi Niv. 4 & 2\\\hline
        Ærkemagi & 3\\\hline
    \end{tabular}
\end{table}
\section*{Evne beskrivelse}
\addcontentsline{toc}{section}{Evne beskrivelse}

\input{../Evne-Ordbog/Ekstra Mana/Ekstra Mana Niv. 3.tex}

\subsection{Lav Manakrystal}
Du kan bruge 15 min på at lave en manakrystal. Dette koster 2 mana, og der er 1 mana i krystallen. Dette skal rollespiles. 


\subsection{Troldmandsmagi Niv. 4}
Troldmanden kan nu kaste niveau 4 magier fra deres sti. Se mere information under kapitlet 'Magi som Troldmand' under sektionen 'Niveauerne'. \\
Derudover får du en ny titel som afhænger af hvilken sti du har valgt.\\
\begin{itemize}
    \item Dæmonologen får titlen \textbf{Jarl af Dæmoner}
    \item Elementalisten får titlen \textbf{Elementernes hersker}
    \item Mentalisten får titlen \textbf{Guru}
    \item Nekromantikeren får titlen \textbf{Livets Vogter}
\end{itemize}

\subsection{Ærkemagi}
Ærkemageren lærer 1 ny niveau 2 magi og 1 ny niveau 3 inden for deres gren af magi (Nekromantiker, Elementalist, Dæmonolog, ovs.)