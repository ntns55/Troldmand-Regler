\chapter{Magi}

\section*{Generelle regler om Magi}
\addcontentsline{toc}{section}{Generelle regler om Magi}
Disse regler gælder alle magier på alle tidspunkter hos alle professioner og der findes ingen undtagelser for normale spillere. Hvis der kommer monstre/plotkarakterer i spil, der kan det medføre en overtrædelse reglerne, men dette vil altid blive nævnt ved briefing.\\
\subsection*{Magiens elve bud}
\addcontentsline{toc}{subsection}{Magiens elve bud}
\begin{enumerate}
    \item Alle pegemagier og områdemagier har en maksimal rækkevidde på 5 meter.
    \item Der findes ingen magi der ikke bruger verbale komponenter, dette er ofte i form af en bøn eller en rite.
    \item Riter skal altid siges højt nok til, at dit mål kan høre det. Hører målet ikke riten, har de ret til at ignorere magien.
    \item Magiens kommando skal altid siges højt, så målet kan høre dig.
    \item Alle magier koster det dobbelte af deres niveau i mana at kaste. Dvs. en niveau 1 magi koster 2 mana, imens en niveau 3 magi koster 6 mana.
    \item Antallet af mana en magibruger har, er det samme som hans optjente XP, en magibruger kan maksimalt opnå 18 mana fra hans XP, samt eventuelt tilkøbt mana, skaffet gennem evner eller lignende. 
    \item Når en magibruger går på 0 LP, går han også automatisk på 0 mana.
    \item Hvis en magibruger kaster en magi, kan de kun holde den tilbage i 2 sekunder, før de skal kaste den. Kaster de den ikke, vil den ramme dem selv uanset effekt eller hvilke barrierer, der måtte beskytte dem fra magi.
    \item Alle magier kan bruges i og udenfor kamp. Vær opmærksom på, at når du forklarer en effekt til et offer, at i begge stadig in-game og derved kan tage skade osv. Hvis du er udenfor kamp eller skal kaste en magi på en person, som ligger ned, er det tilladt at bøje sig over personen og sige effekten lavt, således at du ikke forstyrre spillet.
    \item Alle magier går direkte igennem skjold, rustning og våben.
    \item Bruger du mere end 20 mana bør du tage kontakt til en arrangør.
\end{enumerate}

Der findes grundlæggende fire typer af magier:
\begin{itemize}
    \item Negativ magi
    \item Positiv magi
    \item Øjeblikkelig magi
    \item Passiv magi
\end{itemize}

{\large\textbf{Negativ magi}}\\
Magien påfører en negativ effekt på offeret, som varer i længere tid. En spiller kan kun være påvirket af en negativ magi af gangen. Skulle en spiller der allerede er påvirket af en negativ magi blive ramt af endnu en negativ magi, har den sidste magi ingen effekt.

{\large\textbf{Positiv magi}}\\
Magien påvfører en positiv effekt på en spiller, dette kan være i form af mere LP i længere tid eller et skjold. Skulle en spiller der allerede er påvirket af en positiv magi blive ramt af endnu en positiv magi, har den sidste magi ingen effekt.

{\large\textbf{Øjeblikkelig magi}}\\
Dette er magier der kun vare et øjeblik. Dette kan fx være helbredende magier eller magier der giver skade. Du kan være påvirket af uendelig mange øjeblikkelige magier, da disse vare i under et sekund.

{\large\textbf{Passiv magi}}\\
Magiske effekter som ikke kan fjernes. De vil altid være i effekt og koster ikke mana at bruge. De er også gældende hvis du er død.

Derudover findes der fire kategorier af magier:\\
{\large\textbf{Berøringsmagi}}\\
Når du kaster denne type magi skal du berører dit subjekt.

{\large\textbf{Pegemagi}}\\
Når du kaster denne type magi skal du pege på dit subjekt, der ikke må være længere væk end 5 meter.

{\large\textbf{Områdemagi}}\\
Denne magi er markeret med gryn (f.eks. havregryn). Denne magi aktiver når et eller flere subjekter krydser grynet.

{\large\textbf{Kastemagi}}\\
Denne magi kræver at du har en lille bold, rispose eller lignende. Denne skal du kaste på den du gerne vil ramme. Hvis du rammer med denne rispose så rammer magien også. Vær opmærksom på at selvom der ikke er nogen begrænsning på hvor langt du kan kaste denne bold, skal subjektet stadig kunne hører hvad din magi gør.

\subsection*{At kaste magi}
\addcontentsline{toc}{subsection}{At kaste magi}
Det vigtigste ved magikastning er at være velforberedt. Nogle magier har krav om visse ingredienser, som magikasteren selv skal medbringe. Hvis der er tale om en magi, hvor du skal ramme offeret med en genstand, f.eks. en skumbold, skal du sige effekten når genstanden rammer personen. Hvis genstanden rammer et sværd, skjold eller lignende, skal effekten stadig siges, da magien stadig har sin fulde effekt. Det er kun muligt at undgå magier, hvis offeret skal rammes, flytter sig. Dette kan f.eks. ikke lade sig gøre ved en pegemagi.


\chapter{Magi som Troldmand}

\section{Niveauerne}
Der findes 5 stier indenfor hver gren af magi. Hver gang må får evnen Troldmands magi vil man få adgang til det niveau af magier. Alle Troldmænd får magierne fra den sti som hedder 'Primær Magi'. Dette repræsentere den grundlæggende viden indenfor den gren af magi man har valgt, og derfor er det noget alle lærer.\\
De andre 4 stier skal man dog vælges fra. I niveau 1 får man alle magier. I niveau 2 skal man vælge 3 magier som man gerne vil have. I niveau 3 skal man vælge 2 magier fra de stier man valgte i niveau 2, og i niveau 4 vælger man 1 magi fra de stier man valgte i niveau 3. \\
\textit{\textbf{Eksempel:} Naztur er en Nekromantiker. Når hun får Troldmands magi niv. 1 får hun adgang til "Magisk beskyttelse", "Fjern Zombie", "Feber", "Glemsel" og "Dødens Sandhed".\\ Naztur får nu Troldmands magi niv. 2, og skal vælge en sti fra. Hun må ikke vælge Primær magi. Hun vælger ikke at fokusere på 'Zombie' stien. Derfor får hun magierne: "Skab Zombie"\footnote{Denne er fra primær magi for Nekromantikere}, "Forrådnelse", "Paralyse" og "Søvn".\\ Når Naztur får evnen Troldmandsmagi niv. 3 så skal hun vælge endnu en sti fra. Her vælger hun ikke at tage stien 'Død', da hun stadig ikke må vælge primær magi fra. Hun får magierne: "Ophæv magi", "Spedalsk" og "Sjælebånd".\\
Når Naztur endelig når Troldmands magi niv. 4, skal hun vælge endnu en sti fra, som ikke kan være primær magi. Hun vælger ikke at tage stien 'Sjæl' og får derfor får hun magierne: "Magisk Skjold" og "Pest".}


\section{Magibog og Riter}
Til dine magier er der også knyttet en rite og nogle håndtegn. Denne kombination er helt unik for hver magi, så der ikke kan opstå tvivl om hvilken magi, der bliver kastet. Den enkelte rite skal siges, samtidig med at de tilhørende håndtegn udføres før magien kan kastes. Listen med alle riter kan ses i sektionen "Liste af Riter" til sidst i regelsættet. Hvis man bliver afbrudt i ens riter, skal man starte helt forfra. Magien vil ikke blive brugt og derfor koster forsøget ikke noget mana.\\
Alle magier du kan kaste skal stå i en magi bog. Du kan kun kaste magier når du har din magibog. Hvis du har mistet denne ved f.eks. et tyveri, er du ude af stand til at kaste magier. Ved dette menes der, at du heller ikke kan kaste magier fra andres magi bøger eller hvis du kan en magi udenad. Det er selvfølgelig tilladt at kaste magier som du kan udenad, så længe du har din magi bog på dig. En troldmand må gerne have flere bøger på dig, men kun en bog kan være den rigtige magibog. Når dette er bestemt kan det ikke laves om.


\section{Mana}
Man kan regenerere ens mana ved at meditere. For hvert minut man mediterer får troldmanden 1
mana.\\
\textit{\textbf{Eksempel:} Naztur har 2 mana tilbage. Hun mediterer 16 minutter og genvinder derfor 16 mana. Da Naztur har et maks mana på 14 vil hun nu være på 14 mana.}\\
Når man mediterer skal man sidde helt stille med lukkede øjne og ikke ænse verden omkring sig. Det
vil sige at hvis en mand råber "Se, en troldmand! Lad os dræbe ham og tage alle hans ting!" mens du mediterer, kan du ikke høre det. Du er helt lukket af for verden og er derfor også meget sårbar. Angribes du under meditation vil du stadig få mana for den tid du har mediteret dvs. at hvis du har mediteret i 5 min og bliver angrebet får du 5 mana. Tager du skade mens du mediterer, vågner du. Husk at du går på 0 mana når du går på 0 LP.


\section{Dedikeret mana}
Dedikeret mana er en bestemt måde at bruge sin mana på. Når en magi med denne mana type bliver kastet så mister troldmanden ikke kun mana, men også \textbf{maksimal mana}.\\ Hver gang en magi med denne type bliver brugt \textbf{skal} modtager, mana pris og effekt skrives ned i troldmandens bog. Magier som bruger dedikeret mana kan ikke bruges til at lave skriftruller.\\

\textit{Foreksempel. Tim er en troldmand, med 10 mana til rådighed og 16 mana i alt, som vælger at kaste magien "Følelsesløs" på sin følger: Bjarne den Grusomme. Han skriver først modtagerens navn ned (Bjarne den Grusomme), derefter prisen (4 mana) og sidst effekten (+2 midlertidige LP).\\
Tim har nu kun 6 mana til rådighed at kaste magi for, men den maksimale mængde mana han kan få er 12.}\\

Vær opmærksom på at effekter kastet med dedikeret mana automatisk slutter ved spilstop, eller ved midnat, medmindre andet er beskrevet. Det er derfor vigtigt at du fortæller modtageren af magien hvor lang tid de har denne magi. Når effekten stopper naturligt vil Troldmanden få sin mana tilbage.\\
Hvis du gerne vil fjerne en magi der bruger dedikeret mana fra et subjekt som du har kastet magi på skal dette gøres på følgende måde: En ring a gryn på 1 meter i radius skal laves. Både Troldmanden og subjektet skal stå i cirklen. En hånd placeres på personens pande hvorefter riten: "Trække Magi Dæmon" siges højt.
Derefter skal spilleren informeres om hvilken magi ikke længere påvirker dem, hvis andre magier stadig påvirker dem skal de også informeres om dette. \\
En magi med Dedikeret mana kan fjernes med Ophæv magi, medmindre andet står i magien. Hvis en magi der blev kastet med dedikeret mana bliver ophævet på andre måder end ved brug af ritualet eller at tiden udløber, så vil magikasteren ikke få sin mana tilbage før næste spilgang.\\
Dedikeret mana referer til den mængde mana du ikke har adgang til.

\chapter{Dæmonolog}
Dæmonlogen påkalder utrolige kræfter gennem Raffael moordets domæne. Her er man i stand til ikke blot at snakke med dæmoner, men skabe effekter der varer i længere tid. Dæmonologen specialisere sig i permanente effekter og at lave aftaler. De har også en mulighed for at gøre sig selv mere dæmoniske for at opnå magt. Dette anses som at være en ulovlig form for magi af imperiet.

\begin{table}[H]
    \centering
    \begin{tabular}{|p{0.10\textwidth}|p{0.15\textwidth}|p{0.15\textwidth}|p{0.15\textwidth}|p{0.15\textwidth}|p{0.15\textwidth}|}
    \rowcolor{cerulean!80}\hline
        Niveau & Primær magi & Sælg din sjæl & Dæmonisk hævn & Korruption & Falden Engel \\\hline
        
        1 &         Magisk beskyttelse &        Dæmonisk Kontrakt &         Gnist &         Villigt sind &         Dæmonisk beskyttelse\\\hline        
        2 &         Ophæv magi &                Sandhedens byrde &          Smerte &        Spred korruption &     Følelsesløs\\\hline        
        3 &         Dæmonisk mana &             Det glemte sejl     &       Helvedesild&    Åbent Sind&            Indgyd fanatik\\\hline        
        4 &         Magisk skjold &             Det med småt        &       Flammehav &     Dæmonisk magt &       Bundet sjæl\\\hline
    \end{tabular}
\end{table}

\section{Primær magi}

\begin{primærMagi*}[Magisk Beskyttelse]
\textbf{Type:} Positiv magi\\
\textbf{Kategori:} Berøringsmagi\\
\textbf{Rite:} Magi, Give, Magi\\
\textbf{Effekt:} Subjektet bliver immun overfor den næste negative magi. Denne magi forsvinder, når den er brugt eller efter 30 min.\\
\textbf{Kommando:} Forklar effekten grundigt
\end{primærMagi*}

\begin{primærMagi*}[Ophæv magi]
\textbf{Type:} Øjeblikkelig magi\\
\textbf{Kategori:} Pegemagi\\
\textbf{Rite:} Magi, Trække, Magi\\
\textbf{Effekt:} Ophæver en magisk effekt. Den kan ikke ophæve permanente magiske effekter.\\ 
\textbf{Kommando:} "Ophæv magi"
\end{primærMagi*}

\begin{primærMagi*}[Dæmonisk Mana]
\textbf{Type:} - \\
\textbf{Kategori:} Passiv \\
\textbf{Effekt:} Alle magier der bruger dedikeret mana må nu have to mål i stedet for 1.\\
\textbf{Eksempel:} Tim er en troldmand, med 10 mana til rådighed og 16 mana i alt, som vælger at kaste magien "Følelsesløs" på sine følgere: Bjarne den Grusomme og Mitrildrengen.\\ 
Han skriver først modtagerens navn ned (Bjarne den Grusomme og Mitrildrengen), derefter prisen (4 mana) og sidst effekten (+2 midlertidige LP).\\
Tim har nu kun 6 mana til rådighed at kaste magi for, men den maksimale mængde mana han kan få er 12.
\end{primærMagi*}

\begin{primærMagi*}[Magisk skjold]
\textbf{Type:} -\\
\textbf{Kategori:} Områdemagi \\
\textbf{Rite:} Magi, Give, Liv \\
\textbf{Effekt:} Med denne magi laver troldmanden en magisk cirkel som skal tegnes op med gryn. Cirklen må have
en radius på op til 2,5 meter. Denne kan ingen gå igennem og ingen fysisk skade kan gå igennem. Dog kan magi stadig gå igennem. Skjoldet kan også ophæves med ophæv magi. Skjoldet varer maksimum 10 minutter, eller til kasteren forlader det. \\
\textbf{Note:} Denne magi kan bruges med Dedikeret mana og den vil vare indtil den dedikerede mana trækkes tilbage. Ophæves eller brydes magien vil du ikke få din mana tilbage før næste spilgang.\\
\textbf{Kommando:} "Magisk skjold"
\end{primærMagi*}

\section{Sælg din sjæl}

\begin{sjæl*}[Dæmonisk Kontrakt]
\textbf{Denne magi bruger dedikeret mana.}\\
\textbf{Type:} - \\
\textbf{Kategori:} Speciel\\
\textbf{Rite:} En bøn/ritual på minimum 25 ord til en dæmon.\\
\textbf{Effekt:} Begge parter underskriver en kontrakt. Hvis en af parterne bryderne denne kontrakt så vil denne part dø med det samme. En person kan ikke tvinges til at skrive under på denne kontrakt hverken med magi eller vold. Kontrakten skal være et fysisk dokument som underskrives. Ødelægges kontrakten stopper magien, men Dæmonologen får ikke sin mana tilbage før han har udført ritualet for at tage dedikeret mana tilbage. Ritualet skal gøres på minimum en del af dokumentet.\\
\textbf{Vigtig:} Denne magi bliver ikke ophævet ved spilstop, men mana fra Dedikeret mana gives tilbage.\\
\textbf{Kommando:} Forklar effekten grundigt.
\end{sjæl*}

\begin{sjæl*}[Sandhedens byrde]
\textbf{Denne magi benytter Dedikeret mana.}\\
\textbf{Type:} Passiv \\
\textbf{Kategori:} -\\
\textbf{Effekt:} Du må modificere en kontrakt du har skrevet så kontrakt subjektet kun kan tale sandt. Dette ophæver eller fjerner ikke den originale kontrakt. Dette koster ikke ekstra dedikeret mana ved skabelse. Dette vil varer en spilgang, hvorefter du kan bruge dedikeret mana på at genaktivere effekten. Dette kræver et ritual på 1 minut hvor kontrakt subjektet er til stede.  Denne aktivering vil koste dedikeret mana. Denne effekt kan kun aktiveres med en aktiv kontrakt.\\
\textbf{Kommando:} Forklar effekten grundigt (Husk at magien skal noteres med subjektets navn før magien er gyldig.)
\end{sjæl*}

\begin{sjæl*}[Det glemte sejl]
\textbf{Type:} Passiv \\
\textbf{Kategori:} -\\
\textbf{Effekt:} Du må modificere en kontrakt du har skrevet så kontrakt subjektet vil glemme et element du vælger.  Dette ophæver eller fjerner ikke den originale kontrakt. Dette koster ikke ekstra dedikeret mana ved skabelse. Du kan ikke få dem til at glemme noget om kontrakten. \\
\textbf{Kommando:} Forklar effekten grundigt (Husk at magien skal noteres med subjektets navn før magien er gyldig.)
\end{sjæl*}

\begin{sjæl*}[Det med småt]
\textbf{Denne magi benytter Dedikeret mana.}\\
\textbf{Type:} Passiv \\
\textbf{Kategori:} -\\
\textbf{Effekt:} Du må modificere en kontrakt du har skrevet så kontrakt subjektet kun skal adlyde tre kommandoer du giver denne spilgang.  Dette ophæver eller fjerner ikke den originale kontrakt. Disse kommandoer skal gives med formen: "Kommando: [X]", hvor X er et ord. Dette koster ikke ekstra dedikeret mana ved skabelse af kontrakten. Dette vil varer en spilgang, hvorefter du kan bruge dedikeret mana på at genaktivere effekten med et ritual på 1 minut, hvor subjektet skal være tilstede. Denne aktivering vil koste dedikeret mana. Denne effekt kan kun bruges 1 gang per spilgang. Denne effekt kan kun aktiveres hvis der er en aktiv kontrakt.\\
\textbf{Kommando:} Forklar effekten grundigt (Husk at magien skal noteres med subjektets navn før magien er gyldig.)
\end{sjæl*}




\section{Dæmonisk hævn}

\begin{dHævn*}[Gnist]
\textbf{Type:} Øjeblikkeligmagi\\
\textbf{Kategori:} Kastemagi\\
\textbf{Rite:} Ild, give.\\
\textbf{Effekt:} En rød rispose eller bold kastes på en person. Rammer bolden vil den give 2 skade.\\
\textbf{Kommando:} "Gnist - 2 skade"
\end{dHævn*}

\begin{dHævn*}[Smerte]
\textbf{Type:} Negativ magi\\
\textbf{Kategori:} Berøringsmagi\\
\textbf{Rite:} Magi, Give, Dæmon\\
\textbf{Effekt:} Subjektet bliver fyldt med smerte i 30 sekunder.\\
\textbf{Kommando:} "Smerte - 30 sekunder."
\end{dHævn*}

\begin{dHævn*}[Helvedes ild]
\textbf{Type:} Øjeblikkelig magi\\
\textbf{Kategori:} Berøringsmagi\\
\textbf{Rite:} Dæmon, Ild, Give, Død\\
\textbf{Effekt:} Du bringer helvede op til Kalish for at brænde din fjende. Kraften er afhængig af hvor meget magi du har givet til dæmonerne. En person tager skade af denne magi.\\
\textbf{Kommando:} "Helvedesild - x skade" (Hvor x er mængden af maks mana du har dedikeret gennem Dedikeret mana).
\end{dHævn*}

\begin{dHævn*}[Flammehav]
\textbf{Type:} Øjeblikkelig magi\\
\textbf{Kategori:} Pegemagi\\
\textbf{Rite:} Ild, Give, Give, Ild, Magi\\
\textbf{Effekt:} Op til 5 personer tager skade fra din ild. Du må ikke vælge den samme person flere gange.\\
\textbf{Kommando:} "Flammehav - 3 skade"
\end{dHævn*}

\section{Korruption}
\begin{korruption*}[Villigt sind]
\textbf{Type:} Passiv\\
\textbf{Kategori:} -\\
    \textbf{Effekt:} Du dedikere dig til en dæmon. Vælg en dæmon som du får hjælp af og få denne effekt så længe du opfylder kravene for denne.
        \begin{itemize}
            \item \textbf{Hadets Dæmon:} Smerte ved berøring en gang i timen uden at bruge mana. De relevanter riter eller en bøn på 30 ord skal bruges. Kommando der bruger: "Smerte - 30 Sekunder" - \textbf{krav:} Du skal have en dæmonisk hånd.
            \item \textbf{Grådighedens Dæmon:} Få dobbelt effekt, bonuser og tid, af alle drikke eller andre spiste genstande (Inklusiv dobbelt skade fra gift)
            \item \textbf{Nattens Dæmon:} Genvind 2 mana hver gang du dræber en person - \textbf{krav:} Størknet blod som tårer på kinderne.
            \item \textbf{Jalousiens Dæmon:} Når du torturere en person kan du stjæle 1 mana per minut, hvis dit offer stadig har mana tilbage - \textbf{Krav:} Du skal bære horn.
            \item \textbf{Smertens og Vanviddets Dæmon:} Du får en negativ berørings magi der tager en bøn til Smertens og Vanviddets Dæmon på 50 ord at kaste. Den koster intet mana. Personen vil blive ramt af "Smertens Mærke" som varer indtil den bliver ophævet, indtil de dør eller til slutningen af spilgangen. Folk påvirket af "Smertens Mærke" vil tage 1 ekstra skade fra alle kilder og dobbelt skade fra hellig skade. Kommando: Forklar effekten grundigt. - \textbf{Krav:} Du skal have tydelige ar på kroppen og vil tage dobbelt skade fra hellig skade.
            \item \textbf{Kaosets Dæmon:} Alle magier påvirker dig kun halvt så lang tid. - \textbf{Krav:} Du må ikke bære metal (Inklusiv mønter og nibs)
        \end{itemize}
    \end{korruption*}

\begin{korruption*}[Spred korruption]
\textbf{Denne magi bruger dedikeret mana.}\\
\textbf{Type:} Positivmagi\\
\textbf{Kategori:}Berøringsmagi\\
\textbf{Rite:} En bøn/ritual på 25 ord til en dæmon.\\
\textbf{Effekt:} Giv en af effekterne fra Villigt sind til en anden spiller. De skal stadig opfylde de kravene for velsignelsen\\
\textbf{Kommando:} Forklar effekten grundigt.
\end{korruption*}

\begin{korruption*}[Åbent Sind]
\textbf{Type:} Passiv\\
\textbf{Kategori:} -\\
\textbf{Effekt:} Vælg 2 effekter fra magien Villigt sind.\\
1 af disse effekter kan også være hjælp fra Lysets Dæmon:
\begin{itemize}
    \item \textbf{Lysets Dæmon:} Alle skadende magier du har helbreder i stedet. Alle Helbredende magier skader op til 4 skade. - \textbf{Krav:} Du skal bære et symbol for 2 forskellige guder.
\end{itemize}
\end{korruption*}

\begin{korruption*}[Dæmonisk magt]
\textbf{Type:} Passiv\\
\textbf{Kategori:} -\\
\textbf{Effekt:} Vælg 1 effekt fra magien "Villigt Sind" eller effekten fra Lysets Dæmon fra "Åbent Sind".\\
Du mister evnen ‘Solid mana’.\\ 
Du skal kun betale halv mana pris når du bruger dedikeret mana.
\end{korruption*}

\section{Falden Engel}

\begin{falden*}[Dæmonisk Beskyttelse]
\textbf{Denne magi bruger dedikeret mana.}\\
\textbf{Type:} Positivmagi\\ 
\textbf{Kategori:} Berøringsmagi\\
\textbf{Rite:} En bøn/ritual på 25 ord til en dæmon med et blods offer.\\
\textbf{Effekt:} En spiller bliver immun overfor 1 negativ magi i timen.\\
\textbf{Restriktioner:} Må ikke bruges på magikastere.\\
\textbf{Kommando:} Forklar effekten grundig
\end{falden*}

\begin{falden*}[Følelsesløs]
\textbf{Denne magi bruger dedikeret mana.}\\
\textbf{Type:} Positivmagi\\
\textbf{Kategori:} Berøringsmagi\\
\textbf{Rite:} En bøn/ritual på 25 ord til en dæmon.\\
\textbf{Effekt:} En spiller får +2 midlertidige LP. Disse er de første de mister og men de kan genvindes med naturlig helbredelse eller ved at blive helbredt. Disse tæller mod det maksimale antal midlertidige LP de kan få.\\
\textbf{Restriktioner:} Må ikke bruges på magikastere.\\
\textbf{Kommando:} Forklar effekten grundigt.
\end{falden*}
%\begin{falden*}[Glemslens beskyttelse]
%\textbf{Denne magi benytter Dedikeret mana.}\\
%\textbf{Type:} Positivmagi\\
%\textbf{Kategori:}Berøringsmagi\\
%\textbf{Rite:} Bøn/ritual på 15 ord til en dæmon.\\
%\textbf{Effekt:} Subjektet bliver immun overfor alle glemsels effekter, inklusiv dem fra død.\\
%\textbf{Kommando:} Forklar effekten grundigt.
%\end{falden*}

\begin{falden*}[Indgyd Fanatik]
\textbf{Type:} Positivmagi\\
\textbf{Kategori:} Områdemagi\\
\textbf{Rite:} En tale på minimum 50 ord.\\
\textbf{Effekt:} Alle der har indgået en dæmonisk kontrakt med dig og kan se dig under kamp har +2 Midlertidige LP. Disse LP er de første de mister og kan ikke genvindes før denne magi kastes igen. Denne magi varer 15 minutter eller til de midlertidige LP er mistet.\\
\textbf{Kommando:} Forklar effekt grundigt til alle der bliver påvirket af den.
\end{falden*}

\begin{falden*}[Bundet sjæl]
Denne magi kan kun kastes ved spilstart medmindre det bliver godkendt af en arrangør.\\
Ved spilstart skal dæmonen's udklædning godkendes.\\
\textbf{Denne magi bruger dedikeret mana, men tæller som niveau 2}\\
\textbf{Effekt:} En anden spiller vil skal være klædt ud som dæmon. Denne spiller for lov til at spille din dæmon resten af spilgangen. Denne mana kan ikke trækkes tilbage når den er brugt.
Dæmonen har:\\
\textbf{LP:} 18\\
\textbf{NK:} 15\\
Kan benytte rustning, med Maks RP på 11.\\
Skal være udklædt som dæmon.\\
Når dæmonen har fuld LP vil hans RP begynde at regenerer som ved naturlig helbredelse.\\ 
Dæmonen genvinder 1 LP eller 1 RP hvert minut udenfor kamp og er Immun overfor alle ikke skadende magier dette Inkludere healing og andre boost.\\
Dæmonen tager trippel skade fra hellige våben.\\
\textbf{Kommando:} Forklar effekten MEGET grundigt.
\end{falden*}
\chapter{Elementalist}

En Elementalist har specialiseret sig at bruge naturen omkring ham til at knuse sine fjender. Dog er Elementalisten ikke så brugbar udenfor kamp. Elementalisten vil finde at de vil kunne tage en til to krigere i kamp uden problemer. De vil dog have store bekymringer hvis de skal imod en bueskytte eller en person som bruger pistol. 

\begin{table}[H]
    \centering
    \begin{tabular}{|p{0.10\textwidth}|p{0.15\textwidth}|p{0.15\textwidth}|p{0.15\textwidth}|p{0.15\textwidth}|p{0.15\textwidth}|}
    \rowcolor{cerulean!80}\hline
        Niveau & Primær magi & Vand & Ild & Jord & Vind \\\hline
        
        1 & 
        Magiens sværd & 
        Isslag & 
        Gnist & 
        Mudder maske  & 
        Vindstød\\\hline
        
        2 & 
        Magisk beskyttelse & 
        Hjerne frys & 
        Lyn & 
        Jordens Rustning & 
        Tornado\\\hline
        
        3 & 
        Ophæv magi & 
        Skoldning & 
        Flammehav & 
        Jordens barrier & 
        Vindslid\\\hline
        
        4 & 
        Mana cirkulering & 
        Snestorm & 
        Ildkugle & 
        Lerform & 
        Tordenskrald\\\hline
    \end{tabular}
\end{table}

\section{Primær magi}

\begin{primærMagi*}[Magiens våben]
\textbf{Type:} -\\
\textbf{Kategori:} Passiv\\
\textbf{Effekt:} Tillader Elementalisten at bruge et hvilket som helst våben, dog ikke spyd.
\end{primærMagi*}

\begin{primærMagi*}[Magisk Beskyttelse]
\textbf{Type:} Positiv magi\\
\textbf{Kategori:} Berøringsmagi\\
\textbf{Rite:} Magi, Give, Magi\\
\textbf{Effekt:} Subjektet bliver immun overfor den næste negative magi. Denne magi forsvinder når den er brugt eller efter 30 min.\\
\textbf{Kommando:} Forklar effekten grundigt
\end{primærMagi*}

\begin{primærMagi*}[Ophæv magi]
\textbf{Type:} Øjeblikkelig magi\\
\textbf{Kategori:} Pegemagi\\
\textbf{Rite:} Magi, Trække, Magi\\
\textbf{Effekt:} Ophæver en magisk effekt. Den kan ikke ophæve permanente magiske effekter.\\ 
\textbf{Kommando:} "Ophæv magi"
\end{primærMagi*}

\begin{primærMagi*}[Mana cirkulering]
\textbf{Type:} -\\
\textbf{Kategori:} Passiv\\
\textbf{Effekt:} Når du dræber en person vil du genvinde 2 mana. Dette kan ikke overgå din maksimale mana. Du behøver ikke dræbe en person med en magi for at få denne effekt.
\end{primærMagi*}


\section{Vand}
\begin{vand*}[Isslag]
\textbf{Type:} Negativmagi\\
\textbf{Kategori:} Berøringsmagi\\
\textbf{Rite:} Give, Vand, Trække, Bevægelse\\
\textbf{Effekt:} Dækker subjektet i is således at denne ikke kan bevæge sig.\\
\textbf{Kommando:} ”Isslag, paralyse 10 sekunder”
\end{vand*}

\begin{vand*}[Hjerne frys]
\textbf{Type:} Negativ magi\\
\textbf{Kategori:} Kastemagi\\
\textbf{Rite:} Vand, Bevægelse, Trække\\
\textbf{Ingredienser:} En blå skumbold eller rispose.\\
\textbf{Effekt:} Du kaster en skumbold eller rispose i en blå farve. Rammes personen vil de falde i søvn i 30 sekunder.\\
\textbf{Kommando:} ”Søvn, 30 sekunder.”
\end{vand*}

\begin{vand*}[Skoldning]
\textbf{Type:} Negativ magi\\
\textbf{Kategori:} Pegemagi\\
\textbf{Rite:} Ild, Give, Vand, Give\\
\textbf{Effekt:} Ofret du peger på vil mærke stor smerte i 30 sekunder.\\
\textbf{Kommando:} "Smerte, 30 sekunder."
\end{vand*}

\begin{vand*}[Snestorm]
\textbf{Type:} Øjeblikkeligmagi\\
\textbf{Kategori:} Kastemagi\\
\textbf{Rite:} Vand, Give, Trække, Bevægelse\\
\textbf{Ingredienser:} Gryn\\
\textbf{Effekt:} Sender en sky af ekstrem kulde og sne mod dine fjender. Du skal bruge gryn til denne magi. (Ikke mel eller ris, men gryn). Når du kaster magien skal gryn kastes. De de bliver ramt af gryn vil tage skade.\\
\textbf{Kommando:} "Snestorm, 5 i skade"
\end{vand*}

\section{Ild}
\begin{ild*}[Gnist]
\textbf{Type:} Øjeblikkeligmagi\\
\textbf{Kategori:} Berøringsmagi\\
\textbf{Rite:} Give, Ild, Trække, Ild\\
\textbf{Effekt:} Din hånd blusser op i flammer og tillader dig at efterlade dit mærke i din modstander.\\ 
\textbf{Kommando:} "Gnist, 2 i skade"
\end{ild*}

\begin{ild*}[Lyn]
\textbf{Type:} Øjeblikkeligmagi\\
\textbf{Kategori:} Pegemagi\\
\textbf{Rite:} Trække, Magi, Trække, Ild\\
\textbf{Effekt:} Et lyn skyder fra din finger og efterlader din modstander sort som kul.\\
\textbf{Kommando:} "Lyn, 3 i skade"
\end{ild*}

\begin{ild*}[Flammehav]
\textbf{Type:} Øjeblikkeligmagi\\
\textbf{Kategori:} Pegemagi\\
\textbf{Rite:} Ild, Give, Give, Ild, Magi\\
\textbf{Effekt:} Op til 3 personer tager skade fra din ild. Du må ikke vælge den samme person flere gange.\\
\textbf{Kommando:} "Flammehav - 3 skade"
\end{ild*}

\begin{ild*}[Ildkugle]
\textbf{Type:} Øjeblikkeligmagi\\
\textbf{Kategori:} Kastemagi\\
\textbf{Rite:} Magi, Trække, Ild, Dæmon, Ild\\
\textbf{Ingredienser:} En rød pose eller bold\\
\textbf{Effekt:} Du skaber en brændende kugle, der vil efterlade dine fjender med et stort hul i kroppen.\\
\textbf{Kommando:} "Ildkugle 15 i skade"
\end{ild*}

\section{Jord}
\begin{jord*}[Mudder maske]
\textbf{Type:} Positiv magi\\
\textbf{Kategori:} Berøringsmagi\\
\textbf{Rite:} Jord, Liv, Magi, Liv, Give, Jord, Trække, Død. \\
\textbf{Effekt:} Du beskytter dine mest sårbare dele med størknet mudder. Du er immun overfor skade fra den næste pil, medmindre denne er et bagholdsangreb.\\
\textbf{Kommando:} Forklar grundigt effekten til personen.
\end{jord*}

\begin{jord*}[Jordensrustning]
\textbf{Type:} Positiv magi\\
\textbf{Kategori:} Berøringsmagi - Kun troldmanden.\\
\textbf{Rite:} Jord, Give, Jord\\
\textbf{Effekt:} Du beskytter dig selv mod de næste to slag.\\
\textbf{Kommando:} Ingen
\end{jord*}

\begin{jord*}[Jordens barrier]
\textbf{Type:} -\\
\textbf{Kategori:} Områdemagi \\
\textbf{Rite:} Jord, Give, Bevægelse, Give, Magi.\\
\textbf{Effekt:} Du kan tegne en cirkel eller linje med radius eller længe på maksimalt 2 meter. Folk kan ikke træde igennem denne barriere, og pile vil ikke kunne give skade igennem. Pistol skud vil dog stadig give skade.\\
\textbf{Kommando:} "Jordens skjold"
\end{jord*}

\begin{jord*}[Lerform]
\textbf{Type:} Positiv magi\\
\textbf{Kategori:} Kun Troldmanden\\
\textbf{Rite:} Jord, Trække, Bevægelse, Død\\
\textbf{Effekt:} Du drager med denne magi mineralerne i jorden frem og danner et hårdt lag af ler over din krop. Du er ude af stand til at foretage nogle handlinger eller bevæge dig, men til gengæld er du også immun over alt fysisk skade, inklusiv pistolskud. Du kan stadig blive ramt af magier, undtagen magier der tvinger dig til at bevæge eller flytte dig. Så længe magien er aktiv skal du krydse dine hænder foran dit bryst, når denne stilling brydes eller efter der er gået 10 min ophæves effekten alt efter hvad der sker først.
\end{jord*}

\section{Vind}
\begin{vind*}[Vindstød]
\textbf{Type:} Øjeblikkeligmagi\\
\textbf{Kategori:} Pegemagi\\
\textbf{Rite:} Magi, Trække\\
\textbf{Effekt:} Et kraftigt vindstød vælter din modstander.\\ 
\textbf{Kommando:} ”Vindstød vælt”\\
\end{vind*}

\begin{vind*}[Tornado]
\textbf{Type:} Øjeblikkeligmagi\\
\textbf{Kategori:} Pegemagi\\
\textbf{Rite:} Magi, Trække, Give, Liv, Bevægelse\\
\textbf{Effekt:} Du former en tornado som tvinger en modstander til at smide deres våben.\\
\textbf{Kommando:} "Tornado - Smid dit våben"
\end{vind*}

\begin{vind*}[Vindslid]
\textbf{Type:} Øjeblikkeligmagi\\
\textbf{Kategori:} Berøringsmagi\\
\textbf{Rite:} Bevægelse, Give, Bevægelse, Give, Bevægelse, Give.\\
\textbf{Effekt:} Du bruger erosion til at fjerne alt rustning på en modstander.\\
\textbf{Kommando:} "Vindslid - Mist alt Rustning"
\end{vind*}

\begin{vind*}[Tordenskrald]
\textbf{Type:} -\\
\textbf{Kategori:} Passiv\\
Når du bruger vindstød har du mulighed for at denne også giver 3 i skade. Hvis du gør dette vil vindstød koste 2 ekstra mana. Vindstød vil stadig vælte. Du kan stadig kaste vindstød normalt. Se den nye kommando nedenfor.\\
\textbf{Kommando:} "Tordenskrald - 3 i skade, vælt."
\end{vind*}
\chapter{Mentalist}
Mentalisten har specialiseret sig i sindet. Han har stor stolthed i sine evner. Mentalisten befinder sig ofte tæt ved byer således at han kan studere andre væsner. Dette har konsekvensen at han stor set aldrig kommer i kamp og har derfor størst speciale i ikke-kamp relaterede evner. Dette betyder ikke at han er forsvarsløs, da han stadig besidder magt over sindet.
\begin{table}[H]
    \centering
    \begin{tabular}{|p{0.10\textwidth}|p{0.15\textwidth}|p{0.15\textwidth}|p{0.15\textwidth}|p{0.15\textwidth}|p{0.15\textwidth}|}
    \rowcolor{cerulean!80}\hline
        Niveau & Primær magi & Offensiv & Defensiv & Passiv & Kontrol \\\hline
        
        1 & 
        Magisk Beskyttelse & 
        Aggressiv & 
        Sandhed & 
        Passiv berøring& 
        Glemsel\\\hline
        
        2 & 
        Udvidet Magi & 
        Svaghed & 
        Søvn & 
        Læse Tanker& 
        Venskab\\\hline
        
        3 & 
        Ophæv Magi & 
        Smerte & 
        Paralyse & 
        Fred& 
        Ærefrygt\\\hline
        
        4 & 
        Magisk Skjold & 
        Amok & 
        Energiens Skjold & 
        Energi Flod & 
        Kommando\\\hline
    \end{tabular}
\end{table}
\section{Primær magi}

\begin{primærMagi*}[Magisk Beskyttelse]
\textbf{Type:} Positiv magi\\
\textbf{Kategori:} Berøringsmagi\\
\textbf{Rite:} Magi, Give, Magi\\
\textbf{Effekt:} Subjektet bliver immun overfor den næste negative magi. Denne magi forsvinder når den er brugt eller efter 30 min.\\
\textbf{Kommando:} Forklar effekten grundigt
\end{primærMagi*}

\begin{primærMagi*}[Udvidet magi]
\textbf{Type:} - \\
\textbf{Kategori:} Passiv\\
\textbf{Rite:} Magi, Liv, [Riten til den magi du vil kaste]\\
\textbf{Effekt:} Med denne evne kan mentalisten kaste en magi på to personer samtidig, dette kræver dog meget mere energi end normalt. Magien koster 3 gange normalt mana. \\
\textit{\textbf{Eksempel:}} \textit{En niveau 2 magi med denne effekt koster 12 mana at kaste.}\\
\textbf{Kommando:} -
\end{primærMagi*}

\begin{primærMagi*}[Ophæv magi]
\textbf{Type:} Øjeblikkelig magi\\
\textbf{Kategori:} Pegemagi\\
\textbf{Rite:} Magi, Trække, Magi\\
\textbf{Effekt:} Ophæver en magisk effekt. Den kan ikke ophæve permanente magiske effekter.\\
\textbf{Kommando:} "Ophæv magi"
\end{primærMagi*}

\begin{primærMagi*}[Magisk skjold]
\textbf{Type:} -\\
\textbf{Kategori:} Områdemagi \\
\textbf{Rite:} Magi, Give, Liv \\
\textbf{Effekt:} Med denne magi laver troldmanden en magisk cirkel som skal tegnes op med gryn. Cirklen må have
en radius på op til 2,5 meter. Denne kan ingen gå igennem og ingen fysisk skade kan gå igennem. Dog kan magi stadig gå igennem. Skjoldet kan også ophæves med ophæv magi. Skjoldet varer maksimum 10 minutter, eller til kasteren forlader det. \\
\textbf{Kommando:} "Magisk skjold"
\end{primærMagi*}

\section{Offensiv}

\begin{offensiv*}[Aggressiv]
\textbf{Type:} Negativmagi\\
\textbf{Kategori:} Berøringsmagi\\
\textbf{Rite:} Dæmon, Give, Liv, Magi, Give\\ 
\textbf{Effekt:} Personen bliver sky fra mennesker og ønsker ingen social kontakt med andre, afhængig af personen kan denne blive aggressiv overfor andre. Effekten varer i 10 min.\\
\textbf{Kommando:} Forklar effekten grundigt for subjektet.
\end{offensiv*}

\begin{offensiv*}[Svaghed]
\textbf{Type:} Negativmagi\\
\textbf{Kategori:} Berøringsmagi\\ 
\textbf{Rite:} Trække, Liv, Trække\\
\textbf{Effekt:} Med denne magi kan du berøre et subjekt som vil blive ude af stand til at kæmpe, løbe eller på anden måde udføre større fysiske bedrifter. Magien varer i 15 min.\\
\textbf{Kommando:} Forklar effekten grundigt
\end{offensiv*}

\begin{offensiv*}[Smerte]
\textbf{Type:} Negativmagi\\
\textbf{Kategori:} Pegemagi\\
\textbf{Rite:} Magi, Give, Dæmon\\
\textbf{Effekt:} Subjektet bliver fyldt med smerte i 30 sekunder. Denne magi kan også bruges som berøringsmagi.\\
\textbf{Kommando:} "Smerte, 30 sekunder"
\end{offensiv*}

\begin{offensiv*}[Amok]
\textbf{Type:} Negativmagi\\
\textbf{Kategori:} Pegemagi\\
\textbf{Rite:} Trække, Tanke, Give, Dæmon\\
\textbf{Effekt:} Subjektet går amok i 30 sekunder eller indtil subjektet når 0 LP\\
\textbf{Kommando:} "Amok, 30 sekunder"\\
\end{offensiv*}

\section{Defensiv}

\begin{defensiv*}[Sandhed]
\textbf{Type:} Negativmagi\\
\textbf{Kategori:} Pegemagi\\
\textbf{Rite:} Magi, Tanke, Liv\\
\textbf{Effekt:} Personen kan kun snakke sandt i de næste 10 min.\\
\textbf{Kommando:} Forklar effekten grundigt\\
\end{defensiv*}

\begin{defensiv*}[Søvn]
\textbf{Type:} Negativmagi\\
\textbf{Kategori:} Berøringsmagi\\
\textbf{Rite:} Magi, Trække, Tanke\\
\textbf{Effekt:} Subjektet falder i søvn i 30 sekunder\\
\textbf{Kommando:} "Søvn, 30 sekunder"\\
\end{defensiv*}

\begin{defensiv*}[Paralyse]
\textbf{Type:} Negativmagi\\
\textbf{Kategori:} Pegemagi\\
\textbf{Rite:} Død, Give, Magi\\
\textbf{Effekt:} Subjektet bliver ude af stand til at bevæge sig i 30 sekunder.\\
\textbf{Kommando:} "Paralyse, 30 sekunder"\\
\end{defensiv*}

\begin{defensiv*}[Energiens skjold]
\textbf{Type:} Positivmagi\\
\textbf{Kategori:} Berøringsmagi - Kun mentalisten\\
\textbf{Rite:} Magi, Give, Liv, bevægelse, magi, trække\\
\textbf{Effekt:} Du laver et skjold af ren energi omkring dig, hvilket beskytter dig for 1 skade pr 2 mana du bruger på denne magi. Denne magi kan ikke beskytte mod pistol skud. Du kan bruge et frivilligt antal mana på denne magi, antallet af mana skal dog kunne deles med 2.\\
Denne magi varer 10 min.\\
\textbf{Kommando:} "Energiens skjold"
\end{defensiv*}\todo{Ændret til 10 minutter i stedet for 30 minutter... TBH burde der findes en anden magi}

\section{Passiv}

\begin{passiv*}[Passiv berøring]
\textbf{Type:} Negativmagi\\
\textbf{Kategori:} Berøringsmagi\\
\textbf{Rite:} Give, Liv, Trække, Dæmon\\
\textbf{Effekt:} Subjektet bliver ude af stand til at gøre nogle aggressive handlinger eller tage aggressive beslutninger i 5 min. Subjektet må dog godt udøve selvforsvar.\\
\textbf{Kommando:} Forklar effekten grundigt
\end{passiv*}

\begin{passiv*}[Læse tanker]
\textbf{Type:} Øjeblikkeligmagi\\
\textbf{Kategori:} Berøringsmagi\\
\textbf{Rite:} Tanke, Magi, Trække\\
\textbf{Effekt:} Du er i stand til at læse en persons tanker, med dette menes der at du må stille spilleren 3 spørgsmål til spillet off-game (sørg for at i begiver jer væk fra spillet så i ikke forstyrre nogen) disse skal han besvare. Subjektet ved ikke ved ikke at hans tanker er blevet læst.\\
\textbf{Kommando:} Effekten skal forklares grundigt til subjektet
\end{passiv*}

\begin{passiv*}[Fred]
\textbf{Type:} Negativmagi\\
\textbf{Kategori:} Berøringsmagi\\
\textbf{Rite:} Liv, Magi, Give\\
\textbf{Effekt:} Personen kan ikke angribe dig den næste time\\
\textbf{Kommando:} Forklar effekten grundigt
\end{passiv*}

\begin{passiv*}[Energi flod]
\textbf{Type:} Øjeblikkeligmagi\\
\textbf{Kategori:} Berøringsmagi\\
\textbf{Rite:} Magi, Give, Magi, Give, Magi, Bevægelse, Give\\
\textbf{Effekt:} Du lader din energi løbe igennem dit offer. Du bestemmer hvor meget mana du vælger at bruge på magien, dog skader den alt efter hvor meget mana du vælger at bruge på den.\\
\textbf{Eksempel:} \textit{ Jeg vælger at ofre 3 mana på denne magi, og derfor skader den 3 LP.}\\
\textbf{Kommando:} "Energi flod, x i skade" (Hvor x er mængden af mana du bruger)\\
\end{passiv*}

\section{Kontrol}

\begin{kontrol*}[Glemsel]
\textbf{Type:} Øjeblikkelig magi\\
\textbf{Kategori:} Berøringsmagi\\
\textbf{Rite:} Magi, Trække, Tanke, Trække\\
\textbf{Effekt:} Subjektet glemmer hvad der er sket de sidste 10 minutter.\\
\textbf{Kommando:} Forklar effekten grundigt til subjektet.
\end{kontrol*}

\begin{kontrol*}[Venskab]
\textbf{Type:} Negativmagi\\
\textbf{Kategori:} Pegemagi\\
\textbf{Rite:} Trække, Dæmon, Tanke, Magi\\
\textbf{Effekt:} Subjektet bliver din bedste ven i hele verden i 15 min. Personen kan stadig godt være aggressiv over for andre personer end dig, og vil ikke gå mod sine instinkter.\\
\textbf{Kommando:} Forklar effekten grundigt
\end{kontrol*}

\begin{kontrol*}[Ærefrygt]
\textbf{Type:} Negativmagi\\
\textbf{Kategori:} Pegemagi\\  
\textbf{Rite:} Magi, Tanke, Give
\textbf{Effekt:} 2 personer bliver fyldt med frygt for dig og bliver nød til at knæle. Denne magi kan også bruges som berøringsmagi.\\
\textbf{Kommando:} "Ærefrygt, Knæl, 30 sekunder"
\end{kontrol*}\todo{Sat tid på}

\begin{kontrol*}[Kommando]
\textbf{Type:} Øjeblikkelig magi\\
\textbf{Kategori:} Pegemagi\\
\textbf{Rite:} Magi, Trække, Tanke, Give\\
\textbf{Effekt:} Med denne formular kan troldmanden give en person en enkelt ordre. Ordren skal være på max 5 ord som f.eks. "Åben porten" eller "Smid dit våben" etc. \\
Du kan ikke bede dit offer om direkte at skade andre eller sig selv. Magien varer til ordren er udført eller der er gået 30 sekunder, alt efter hvad der
sker først.\\
\textbf{Kommando:} "Kommando: [indsæt ordre]"
\end{kontrol*}
\chapter*{Nekromantiker}
\addcontentsline{toc}{chapter}{Nekromantiker}
Nekromantikere besidder evnen til at manipulerer med livs energi. Dette kan være i form af at give den til de døde eller tage den fra de levende.

\begin{table}[H]
    \centering
    \begin{tabular}{|p{0.10\textwidth}|p{0.15\textwidth}|p{0.15\textwidth}|p{0.15\textwidth}|p{0.15\textwidth}|p{0.15\textwidth}|}
    \rowcolor{cerulean!80}\hline
        Niveau & Primær magi & Zombie & Sygdomens Mørke  & Sjæl & Forrådnelse \\\hline
        
        1 & 
        Fjern Zombie & 
        Skab Zombie & 
        Betændelse & 
        Glemsel& 
        Dødens Gaver\\\hline
        
        2 & 
        Søvn & 
        Helbred Zombie & 
        Feber & 
        Dødens sandhed& 
        Sjælens batteri\\\hline
        
        3 & 
        Ophæv magi & 
        Overfør liv& 
        Spedalsk & 
        Sjælens Pil& 
        Dræn liv\\\hline
        
        4 & 
        Døds cirkel & 
        Skab døds ridder& 
        Pest & 
        Sjælens energi & 
        Dødens Genopstand 
        \\\hline
    \end{tabular}
\end{table}

\section*{Sygdom}\addcontentsline{toc}{section}{Sygdom}
Sygdomme fungerer ikke som almindelig magi. Disse gælder ikke som negative magier ej heller positive. Sygdomme har ingen tidsgrænse men bliver der indtil personen bliver kureret, selv hvis de dør vil de stadig være påvirket.\\
En person kan godt være påvirket af flere sygdomme på en gang. Hvis personen bliver kureret forsvinder alle sygdomme.

\section*{Primær magi}\addcontentsline{toc}{section}{Primær magi}

\begin{primærMagi*}[Fjern Zombie]
\textbf{Type:} Øjeblikkeligmagi\\
\textbf{Kategori:} Pegemagi\\
\textbf{Rite:} Trække, Magi, Trække, Død\\
\textbf{Effekt:} Dræber en zombie.\\
\textbf{Kommando:} "Fjern zombie"
\end{primærMagi*}

\begin{primærMagi*}[Søvn]
\textbf{Type:} Negativmagi\\
\textbf{Kategori:} Berøringsmagi\\
\textbf{Rite:} Magi, Trække, Tanke\\
\textbf{Effekt:} Subjektet falder i søvn i 30 sekunder\\
\textbf{Kommando:} "Søvn, 30 sekunder"\\
\end{primærMagi*}

\begin{primærMagi*}[Ophæv magi]
\textbf{Type:} Øjeblikkelig magi\\
\textbf{Kategori:} Pegemagi\\
\textbf{Rite:} Magi, Trække, Magi\\
\textbf{Effekt:} Ophæver en magisk effekt. Den kan ikke ophæve permanente magiske effekter.\\
\textbf{Kommando:} "Ophæv magi"
\end{primærMagi*}

\begin{primærMagi*}[Døds cirkel]
\textbf{Type:} -\\
\textbf{Kategori:} Områdemagi \\
\textbf{Rite:} Magi, Give, Død, Trække, Liv \\
\textbf{Effekt:} Med denne magi laver troldmanden en magisk cirkel som skal tegnes op med gryn. Cirklen må have
En radius på op til 2,5 meter. Alle der træder ind eller ud af cirklen vil tage 4 i skade. Kasteren skal stå i Cirklen før den har effekt. Hvis en person dør til dette skjold, vil de blive gjort til en zombie, som om Nekromantikeren havde kastet Skab Zombie på dem. Skjoldet kan ophæves med ophæv magi. Skjoldet varer Maksimum 10 minutter, eller til kasteren forlader det.\\ 
\textbf{Kommando:} Døds cirkel 4 i skade (Hvis personen dør, informer dem om zombie reglerne.)
\end{primærMagi*}

\section*{Zombie}\addcontentsline{toc}{section}{Zombie}
\begin{zombie*}[Skab zombie]
\textbf{Type:} Negativmagi\\
\textbf{Kategori:} Berøringsmagi\\
\textbf{Rite:} Død, Magi, Trække, Død\\

\textbf{Effekt:} Med denne magi kan troldmanden skabe en zombie ud af et lig. Personen skal sminkes hvid i hovedet og/eller sort omkring øjnene og gå meget langsomt, samt kun gøre hvad dens skaber har kommanderet.\\
Zombien's liv afhænger af hvor mange magier i zombie stien du har adgang til.\\ Hvis du kun har adgang til niveau 1 magi i Zombie stien har dine zombier 4 LP.\\ Har du adgang til niveau 2 magi i Zombie stien har dine zombier 6 LP.\\ Har du adgang til niveau 3 magi i Zombie stien har dine zombier 8 LP.\\ Har du adgang til niveau 4 magi i Zombie stien har dine zombier 12 LP.\\
En zombie's nævekamp vil være den samme som da de var levende uanset niveau. Denne magi koster ikke mere mana selvom dine zombier har mere liv. Magien varer maksimum 30 minutter. Når en person som har været zombie bliver genoplivet, vil de være normal igen, og kan intet huske af dens liv som zombie. Husk at fjerne sminken. Zombier tager dobbelt skade fra alle hellige våben (markeret med et grønt bånd), dette skal zombien selv være opmærksom på. Zombier kan ikke genvinde LP, medmindre
magien Helbred zombie bliver brugt.\\
\textit{MEGET VIGTIGT: ZOMBIERE KAN IKKE KÆMPE HURTIGT ELLER LØBE HURTIGT, DE SKAL OPFØRE SIG SOM ZOMBIER, OG KÆMPE SOM ZOMBIERE. DETTE ER NEKROMANTIKERENS ANSVAR.}\\
\textbf{Kommando:} Forklar effekten grundigt til subjektet.\\
\end{zombie*}

\begin{zombie*}[Helbred Zombie]
\textbf{Type:} Øjeblikkeligmagi\\
\textbf{Kategori:} Berøringsmagi\\
\textbf{Rite:} Give, Død, Liv, Give, Liv, Død\\
\textbf{Effekt:} En zombie genvinder alle LP.\\
\textbf{Kommando:} Forklar effekten grundigt.
\end{zombie*}

\begin{zombie*}[Overfør liv]
\textbf{Type:} Øjeblikkeligmagi\\
\textbf{Kategori:} Berøringsmagi\\
\textbf{Rite:} Magi, Liv, Død, Magi\\
\textbf{Effekt:} Du kan overføre liv fra et villigt subjekt, eller zombie, til et andet. Ingen af personerne kan overstige maksimum LP eller gå på 0 LP.\\
\textbf{Kommando:} Forklar effekten grundigt
\end{zombie*}

\begin{zombie*}[Skab døds ridder]
\textbf{Type:} Negativmagi\\
\textbf{Kategori:} Berøringsmagi\\
\textbf{Rite:} Trække, Liv, Give, Død, Give, Tanke\\
\textbf{Effekt:} Med denne magi skaber nekromantikeren en døds ridder. Døds ridderen vil have sin egen vilje, men skal adlyde nekromantikeren. Rideren er immun overfor alle magier, inklusiv dem fra Nekromantikeren, dette inkludere ophæv magi. De tager dobbelt skade fra hellige våben. De kan ikke genvinde LP fra magier eller fra naturlig helbredelse. Denne magi varer til døds ridderen dør. Døds Ridderen har 15 LP og 15 NK.\\
Derudover kan de kaste Helbred Zombie 2 gange. Dette har samme effekt og rite som nekromantikeren's helbred zombie, men der behøves ikke nogen magibog.
\textbf{Kommando:} Forklar effekten til begge personer\\

\end{zombie*}

\section*{Sygdommens Mørke}\addcontentsline{toc}{section}{Sygdommens Mørke}

\begin{sygdom*}[Forrådnelse]
\textbf{Type:} Sygdom\\
\textbf{Kategori:} Berøringsmagi\\
\textbf{Rite:} Magi, Give, Dæmon\\
\textbf{Effekt:} Subjektet føler stor smerte, men kan dog stadig kæmpe så længe der gives udtryk for smerten.\\
\textbf{Kommando:} Forklar effekten grundig til subjektet. Husk at forklare hvad en sygdom er.
\end{sygdom*}

\begin{sygdom*}[Feber]
\textbf{Type:} Sygdom\\
\textbf{Kategori:} Berøringsmagi\\
\textbf{Rite:} Give, Død, Magi, Dæmon\\
\textbf{Effekt:} Subjektet mister 2 maks LP og føler sig syg lige så snart han sidder stille. Maks LP kan aldrig være mindre end 1.\\
\textbf{Kommando:} Forklar effekten grundigt til subjektet. Husk at forklare hvad en sygdom er.
\end{sygdom*}

\begin{sygdom*}Spedalsk
\textbf{Type:} Sygdom\\
\textbf{Kategori:} Berøringsmagi\\
\textbf{Rite:} Trække, Liv, Trække\\
\textbf{Effekt:} Med denne magi kan du berøre et subjekt som vil blive ude af stand til at kæmpe, løbe eller på anden måde udføre større fysiske bedrifter.\\
\textbf{Kommando:} Forklar effekten grundigt til subjektet. Husk at forklare hvad en sygdom er.
\end{sygdom*}

\begin{sygdom*}[Pest]
\textbf{Type:} Sygdom\\
\textbf{Kategori:} Berøringsmagi\\
\textbf{Rite:} Give, Død\\
\textbf{Effekt:} Subjektet mister 1 LP hvert minut indtil denne er på 1 LP. Subjektet er ikke i stand til at genvinde liv så længe de er under denne effekt, også selvom det er igennem helbredelse eller magi. Subjektet føler sig syg når de står stille eller sidder.\\
\textbf{Kommando:} Forklar effekten grundigt til subjektet. Husk at forklare hvad en sygdom er.
\end{sygdom*}

\section*{Sjæl}\addcontentsline{toc}{section}{Sjæl}
\begin{nSjæl*}[Glemsel]
\textbf{Type:} Øjeblikkeligmagi\\
\textbf{Kategori:} Berøringsmagi\\
\textbf{Rite:} Magi, Trække, Tanke, Trække\\
\textbf{Effekt:} Får et subjekt til at glemme de sidste 10 minutter.\\
\textbf{Kommando:} Forklar effekten til subjektet
\end{nSjæl*}

\begin{nSjæl*}[Dødens Sandhed]
\textbf{Type:} Øjeblikkeligmagi\\
\textbf{Kategori:} Berøringsmagi\\
\textbf{Rite:} Liv, Død, Liv\\
\textbf{Effekt:} En død person skal svare sandt på 1 spørgsmål.\\
\textbf{Kommando:} Forklar dette grundigt til subjektet
\end{nSjæl*}

\begin{nSjæl*}[Sjælens Pil]
\textbf{Type:} Øjeblikkeligmagi\\
\textbf{Kategori:} Pegemagi\\
\textbf{Rite:} Trække, Liv, Give, Dæmon, Ild\\
\textbf{Effekt:} Nekromantikeren Fylder sine hånd med ren død. Offeret vil tage 3 i skade.\\
\textbf{Kommando:} Sjælens pil 3 i skade.\\

\end{nSjæl*}

\begin{nSjæl*}[Sjælens energi]
\textbf{Type:} Øjeblikkeligmagi\\
\textbf{Kategori:} Berøringsmagi\\
\textbf{Rite:} Trække, Liv, [Riten til den magi du vil kaste], Magi\\
\textbf{Effekt:} Du kan bruge LP som mana enten fra dig selv eller fra et villigt subjekt. For at bruge denne magi skal riterne trække og liv sættes foran den originale rite og riten magi skal sættes efter. Der kan ikke kastes niveau 4 magier med denne magi.
\end{nSjæl*}

\section*{Forrådnelse}\addcontentsline{toc}{section}{Forrådnelse}

\begin{død*}[Dødens gave]
textbf{Type:} Passiv\\
\textbf{Kategori:} -\\
\textbf{Effekt:} Nekromantikeren er immun overfor alle glemsels effekter, inklusiv glemsel fra at være slået ud eller være bevidstløs og død.
\end{død*}

\begin{død*}[Sjælens batteri]
\textbf{Type:} Passiv \\
\textbf{Kategori:} -\\
\textbf{Effekt:} Når nekromantikeren gennemleder et lig for penge vil de også automatisk genvinde 2 Mana. Det tager 30 sekunder at gennemlede en person.\\ 
\end{død*}

\begin{død*}[Dræn liv]
\textbf{Type:} Øjeblikkeligmagi\\
\textbf{Kategori:} Berøringsmagi\
\textbf{Rite:} Trække, Liv, Give, Død, Give, Liv\\
\textbf{Effekt:} Et subjekt tager 5 skade. Nekromantikeren genvinder 5 liv.\\
\textbf{Kommando:} ”Dræn liv, 5 skade”
\end{død*}

\begin{død*}[Dødens genopstand]
\textbf{Type:} - \\
\textbf{Kategori:} Område magi\\
\textbf{Effekt:} Nekromantikeren forbereder et alter, dette skal være tydeligt og skal være dekoreret. Dette alter bliver forberedt ved at udfør et mindre ritual som varer minimum 5 minutter og hvor et levende væsen ofres.\\ 
Imens magien har en effekt vil nekromantikeren genopstå  ved alteret hvis de bliver slået ned. Denne Magi varer indtil du dør. Nekromantikeren lægger en hånd på hovedet og går hen til alteret når de dør. Denne magi tæller ikke som en positiv magi, men som en passiv effekt og kan derfor ikke ophæves med ophæv magi.  Man kan kun genoplive med denne effekt 1 gang per 30 minutter. Hvis alteret er intakt genopstår man med fuldt lp og mana og ignorere døds reglerne. Er alteret blevet forstyrret eller ødelagt vil Nekromantikeren ikke kunne genvinde mana ved meditation i 10 minutter. Det skal være nemt for en anden spiller at forstyrre eller ødelægge alteret. Man må kun have et alter.
\end{død*}