\chapter*{Niveau 2}
\addcontentsline{toc}{chapter}{Niveau 2}
At starte et bål med magi er ikke længere et problem, og du har fundet din gren af magien som vil guide dig igennem verden. Der skal dog ikke meget til at vælte dig omkuld og du skal passe på at du ikke bliver overvældet af folk som ønsker at gøre dig ondt.

\begin{table}[H]
    \centering
    \begin{tabular}{|p{0.50\textwidth}|p{0.25\textwidth}|}
    \rowcolor{cerulean!80}\hline
    Evne navn & Pris i XP \\\hline
        Ekstra Mana Niv. 1 & 1 \\\hline
        Opsug Manakrystal & 3\\\hline
        Troldmandsmagi Niv. 2 & 2\\\hline
    \end{tabular}
\end{table}
\section*{Evne beskrivelse}
\addcontentsline{toc}{section}{Evne beskrivelse}

\subsection*{Ekstra Mana Niv. 1}\addcontentsline{toc}{subsection}{Ekstra Mana Niv. 1} 
Du har nu \textbf{2} ekstra mana.

\subsection*{Opsug Manakrystal}\addcontentsline{toc}{subsection}{Opsug Manakrystal}
Du er nu i stand til at bruge den gemte mana i en manakrystal til at kaste magier. For at få denne mana ud skal ritten “Trække” bruges og du vil genvinde 1 mana. (manakrystalen bliver ødelagt ved brug)

\subsection*{Troldmandsmagi Niv. 2}\addcontentsline{toc}{subsection}{Troldmandsmagi Niv. 2}
Troldmanden kan nu kaste niveau 2 magier fra deres sti. Se mere information under kapitlet 'Magi som Troldmand' under sektionen 'Niveauerne'. \\
Derudover får du en ny titel som afhænger af hvilken sti du har valgt.\\
\begin{itemize}
    \item Arkanisten får titlen \textbf{Novice}
    \item Dæmonologen får titlen \textbf{Dæmon tilbeder}
    \item Elementalisten får titlen \textbf{Vandre}
    \item Mentalisten får titlen \textbf{Discipel}
    \item Nekromantikeren får titlen \textbf{Dødens discipel}
\end{itemize}